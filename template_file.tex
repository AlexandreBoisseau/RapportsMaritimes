\documentclass[12pt,a4paper]{article}
\usepackage[utf8]{inputenc}
\usepackage[T1]{fontenc}
\usepackage{graphicx}
\usepackage{amsmath}
\usepackage{amsfonts}
\usepackage{amssymb}
\usepackage{geometry}
\usepackage{fancyhdr}
\usepackage[francais]{babel}
\usepackage[hidelinks]{hyperref}
\usepackage{booktabs}
\usepackage{enumitem}
\usepackage{hyperref}

\geometry{left=2cm,right=2cm,top=2.5cm,bottom=2.5cm}

\pagestyle{fancy}
\fancyhf{}
\lhead{\textbf{Rapport d'Expertise Maritime}}
\rhead{Date: \today}
\lfoot{XI_MAC_EX}

\title{Rapport d'Expertise Maritime}
\author{XI_MAC_EX}
\date{{\Large XI_DATE_EX}}

\begin{document}
\maketitle
\tableofcontents
\newpage

\section{Introduction}
Ce rapport présente les résultats de l'expertise maritime effectuée sur le navire XI_SHIP_NAME_EX, immatriculé sous le numéro XI_SHIP_IMM_EX, à la demande de ZZZ.

\section{Informations générales}
\begin{itemize}
    \item Nom du navire: XI_SHIP_NAME_EX
    \item Immatriculation: XI_SHIP_IMM_EX
    \item Type de navire: AAA
    \item Port d'attache: BBB
    \item Année de construction: CCC
    \item Chantier naval: DDD
\end{itemize}

\section{Objectif de l'expertise}
L'objectif de cette expertise est de vérifier l'état général du navire, d'identifier les éventuelles réparations nécessaires et d'évaluer la conformité aux réglementations en vigueur.

\section{Méthodologie}
L'expertise a été réalisée selon les étapes suivantes:
\begin{enumerate}
    \item Inspection visuelle de la coque et de la structure
    \item Vérification des systèmes de propulsion et de navigation
    \item Inspection des équipements de sécurité
    \item Vérification de la conformité aux normes environnementales
\end{enumerate}

\newpage
\section{Résultats de l'expertise}
XI_PAGES_EX

\end{document}
